% Options for packages loaded elsewhere
\PassOptionsToPackage{unicode}{hyperref}
\PassOptionsToPackage{hyphens}{url}
\PassOptionsToPackage{dvipsnames,svgnames,x11names}{xcolor}
\documentclass[
  11pt,
]{article}
\usepackage{xcolor}
\usepackage[margin=2cm]{geometry}
\usepackage{amsmath,amssymb}
\setcounter{secnumdepth}{-\maxdimen} % remove section numbering
\usepackage{iftex}
\ifPDFTeX
  \usepackage[T1]{fontenc}
  \usepackage[utf8]{inputenc}
  \usepackage{textcomp} % provide euro and other symbols
\else % if luatex or xetex
  \usepackage{unicode-math} % this also loads fontspec
  \defaultfontfeatures{Scale=MatchLowercase}
  \defaultfontfeatures[\rmfamily]{Ligatures=TeX,Scale=1}
\fi
\usepackage{lmodern}
\ifPDFTeX\else
  % xetex/luatex font selection
  \setmainfont[]{Segoe UI}
  \setmonofont[]{Consolas}
\fi
% Use upquote if available, for straight quotes in verbatim environments
\IfFileExists{upquote.sty}{\usepackage{upquote}}{}
\IfFileExists{microtype.sty}{% use microtype if available
  \usepackage[]{microtype}
  \UseMicrotypeSet[protrusion]{basicmath} % disable protrusion for tt fonts
}{}
\makeatletter
\@ifundefined{KOMAClassName}{% if non-KOMA class
  \IfFileExists{parskip.sty}{%
    \usepackage{parskip}
  }{% else
    \setlength{\parindent}{0pt}
    \setlength{\parskip}{6pt plus 2pt minus 1pt}}
}{% if KOMA class
  \KOMAoptions{parskip=half}}
\makeatother
\usepackage{color}
\usepackage{fancyvrb}
\newcommand{\VerbBar}{|}
\newcommand{\VERB}{\Verb[commandchars=\\\{\}]}
\DefineVerbatimEnvironment{Highlighting}{Verbatim}{commandchars=\\\{\}}
% Add ',fontsize=\small' for more characters per line
\usepackage{framed}
\definecolor{shadecolor}{RGB}{248,248,248}
\newenvironment{Shaded}{\begin{snugshade}}{\end{snugshade}}
\newcommand{\AlertTok}[1]{\textcolor[rgb]{0.94,0.16,0.16}{#1}}
\newcommand{\AnnotationTok}[1]{\textcolor[rgb]{0.56,0.35,0.01}{\textbf{\textit{#1}}}}
\newcommand{\AttributeTok}[1]{\textcolor[rgb]{0.13,0.29,0.53}{#1}}
\newcommand{\BaseNTok}[1]{\textcolor[rgb]{0.00,0.00,0.81}{#1}}
\newcommand{\BuiltInTok}[1]{#1}
\newcommand{\CharTok}[1]{\textcolor[rgb]{0.31,0.60,0.02}{#1}}
\newcommand{\CommentTok}[1]{\textcolor[rgb]{0.56,0.35,0.01}{\textit{#1}}}
\newcommand{\CommentVarTok}[1]{\textcolor[rgb]{0.56,0.35,0.01}{\textbf{\textit{#1}}}}
\newcommand{\ConstantTok}[1]{\textcolor[rgb]{0.56,0.35,0.01}{#1}}
\newcommand{\ControlFlowTok}[1]{\textcolor[rgb]{0.13,0.29,0.53}{\textbf{#1}}}
\newcommand{\DataTypeTok}[1]{\textcolor[rgb]{0.13,0.29,0.53}{#1}}
\newcommand{\DecValTok}[1]{\textcolor[rgb]{0.00,0.00,0.81}{#1}}
\newcommand{\DocumentationTok}[1]{\textcolor[rgb]{0.56,0.35,0.01}{\textbf{\textit{#1}}}}
\newcommand{\ErrorTok}[1]{\textcolor[rgb]{0.64,0.00,0.00}{\textbf{#1}}}
\newcommand{\ExtensionTok}[1]{#1}
\newcommand{\FloatTok}[1]{\textcolor[rgb]{0.00,0.00,0.81}{#1}}
\newcommand{\FunctionTok}[1]{\textcolor[rgb]{0.13,0.29,0.53}{\textbf{#1}}}
\newcommand{\ImportTok}[1]{#1}
\newcommand{\InformationTok}[1]{\textcolor[rgb]{0.56,0.35,0.01}{\textbf{\textit{#1}}}}
\newcommand{\KeywordTok}[1]{\textcolor[rgb]{0.13,0.29,0.53}{\textbf{#1}}}
\newcommand{\NormalTok}[1]{#1}
\newcommand{\OperatorTok}[1]{\textcolor[rgb]{0.81,0.36,0.00}{\textbf{#1}}}
\newcommand{\OtherTok}[1]{\textcolor[rgb]{0.56,0.35,0.01}{#1}}
\newcommand{\PreprocessorTok}[1]{\textcolor[rgb]{0.56,0.35,0.01}{\textit{#1}}}
\newcommand{\RegionMarkerTok}[1]{#1}
\newcommand{\SpecialCharTok}[1]{\textcolor[rgb]{0.81,0.36,0.00}{\textbf{#1}}}
\newcommand{\SpecialStringTok}[1]{\textcolor[rgb]{0.31,0.60,0.02}{#1}}
\newcommand{\StringTok}[1]{\textcolor[rgb]{0.31,0.60,0.02}{#1}}
\newcommand{\VariableTok}[1]{\textcolor[rgb]{0.00,0.00,0.00}{#1}}
\newcommand{\VerbatimStringTok}[1]{\textcolor[rgb]{0.31,0.60,0.02}{#1}}
\newcommand{\WarningTok}[1]{\textcolor[rgb]{0.56,0.35,0.01}{\textbf{\textit{#1}}}}
\usepackage{longtable,booktabs,array}
\usepackage{calc} % for calculating minipage widths
% Correct order of tables after \paragraph or \subparagraph
\usepackage{etoolbox}
\makeatletter
\patchcmd\longtable{\par}{\if@noskipsec\mbox{}\fi\par}{}{}
\makeatother
% Allow footnotes in longtable head/foot
\IfFileExists{footnotehyper.sty}{\usepackage{footnotehyper}}{\usepackage{footnote}}
\makesavenoteenv{longtable}
\setlength{\emergencystretch}{3em} % prevent overfull lines
\providecommand{\tightlist}{%
  \setlength{\itemsep}{0pt}\setlength{\parskip}{0pt}}
\usepackage{bookmark}
\IfFileExists{xurl.sty}{\usepackage{xurl}}{} % add URL line breaks if available
\urlstyle{same}
\hypersetup{
  colorlinks=true,
  linkcolor={Maroon},
  filecolor={Maroon},
  citecolor={Blue},
  urlcolor={Blue},
  pdfcreator={LaTeX via pandoc}}

\author{}
\date{}

\begin{document}

\section{TwinScale-Lite Demo Senaryosu: Akıllı Şehir Deprem İzleme
Sistemi}\label{twinscale-lite-demo-senaryosu-akux131llux131-ux15fehir-deprem-izleme-sistemi}

\subsection{Senaryo Özeti}\label{senaryo-uxf6zeti}

Bir şehirdeki deprem izleme ve erken uyarı sistemini dijital ikiz olarak
modelleyeceğiz. \textbf{3 farklı Thing tipi} (Sensor, Device, Component)
ve \textbf{3 farklı domain} (seismic, environmental, air\_quality)
kullanarak toplamda \textbf{7 Thing} kaydedilecektir.

\begin{center}\rule{0.5\linewidth}{0.5pt}\end{center}

\subsection{Faz 1 — Altyapı Bileşenleri (Component-Based
Things)}\label{faz-1-altyapux131-bileux15fenleri-component-based-things}

\subsubsection{1.1 Building: “Kadıköy
Hastanesi”}\label{building-kadux131kuxf6y-hastanesi}

\begin{longtable}[]{@{}ll@{}}
\toprule\noalign{}
Alan & Değer \\
\midrule\noalign{}
\endhead
\bottomrule\noalign{}
\endlastfoot
\textbf{Thing Type} & \texttt{component} \\
\textbf{DTDL Interface} & \texttt{dtmi:iodt2:Building;1} \\
\textbf{ID} & \texttt{kadikoy-hastanesi} \\
\textbf{Name} & Kadıköy Devlet Hastanesi \\
\textbf{Description} & 7 katlı hastane binası, kritik altyapı \\
\end{longtable}

\textbf{Adımlar:}

\begin{enumerate}
\def\labelenumi{\arabic{enumi}.}
\tightlist
\item
  Sol menüden \textbf{“Create Thing”} sayfasını aç
\item
  Thing Type olarak \textbf{Component} seç
\item
  \textbf{“Select DTDL Interface”} butonuna tıkla → Modal’da
  \texttt{Building} seç
\item
  DTDL özeti görünür: 5 property, 3 telemetry, 2 command
\item
  \textbf{Auto-Fill (Wand)} butonuna tıkla → Tüm alanlar otomatik dolar
\item
  Property’leri doldur:
\end{enumerate}

\begin{longtable}[]{@{}llll@{}}
\toprule\noalign{}
Property & Type & Value/Description & Writable \\
\midrule\noalign{}
\endhead
\bottomrule\noalign{}
\endlastfoot
\texttt{buildingType} & string (Enum) & \texttt{hospital} & No \\
\texttt{floors} & integer & 7 & No \\
\texttt{constructionYear} & integer & 2005 & No \\
\texttt{seismicRating} & string (Enum) & \texttt{high} & Yes \\
\texttt{occupancy} & integer & 350 & Yes \\
\texttt{vibrationLevel} & float & Telemetry — m/s² & No \\
\texttt{structuralHealth} & float & Telemetry — \% & No \\
\texttt{tiltAngle} & float & Telemetry — derece & No \\
\end{longtable}

\begin{enumerate}
\def\labelenumi{\arabic{enumi}.}
\setcounter{enumi}{6}
\tightlist
\item
  \textbf{Commands} sekmesi: \texttt{evacuate}, \texttt{lockdown} zaten
  Auto-Fill ile geldi
\item
  \textbf{Location} sekmesine geç → Haritada Kadıköy’de hastane konumunu
  tıkla

  \begin{itemize}
  \tightlist
  \item
    Koordinatlar otomatik set edilir (ör: \texttt{40.9828,\ 29.0292})
  \item
    Reverse geocoding ile adres otomatik dolar
  \item
    Altitude API’den yükseklik gelir
  \end{itemize}
\item
  Sağ panelde \textbf{YAML Preview} kontrol et — TwinInterface ve
  TwinInstance YAML’ları canlı görünür
\item
  \textbf{DTDL Validation Panel} yeşil tik gösterir
\item
  \textbf{“Create”} butonuna bas → Başarılı kayıt
\end{enumerate}

\begin{center}\rule{0.5\linewidth}{0.5pt}\end{center}

\subsubsection{1.2 Street: “Bağdat Caddesi - Segment
A”}\label{street-baux11fdat-caddesi---segment-a}

\begin{longtable}[]{@{}ll@{}}
\toprule\noalign{}
Alan & Değer \\
\midrule\noalign{}
\endhead
\bottomrule\noalign{}
\endlastfoot
\textbf{Thing Type} & \texttt{component} \\
\textbf{DTDL Interface} & \texttt{dtmi:iodt2:Street;1} \\
\textbf{ID} & \texttt{bagdat-caddesi-seg-a} \\
\textbf{Name} & Bağdat Caddesi - Segment A \\
\textbf{Description} & Ana tahliye güzergahı, 1.2 km \\
\end{longtable}

\textbf{Adımlar:}

\begin{enumerate}
\def\labelenumi{\arabic{enumi}.}
\tightlist
\item
  Create Thing sayfasında Thing Type: \textbf{Component}
\item
  DTDL Interface: \textbf{Street} seç → Auto-Fill uygula
\item
  Property’leri doldur:
\end{enumerate}

\begin{longtable}[]{@{}llll@{}}
\toprule\noalign{}
Property & Type & Value & Writable \\
\midrule\noalign{}
\endhead
\bottomrule\noalign{}
\endlastfoot
\texttt{streetName} & string & Bağdat Caddesi & No \\
\texttt{streetType} & string (Enum) & \texttt{main\_road} & No \\
\texttt{length} & float & 1200.0 & No \\
\texttt{lanes} & integer & 4 & No \\
\texttt{isEvacuationRoute} & boolean & \texttt{true} & Yes \\
\texttt{surfaceDamage} & float & Telemetry — \% & No \\
\texttt{trafficFlow} & float & Telemetry — adet/dk & No \\
\texttt{blockageLevel} & float & Telemetry — \% & No \\
\end{longtable}

\begin{enumerate}
\def\labelenumi{\arabic{enumi}.}
\setcounter{enumi}{3}
\tightlist
\item
  Commands: \texttt{closeStreet}, \texttt{redirectTraffic}
\item
  Location: Bağdat Caddesi üzerinde bir nokta seç
\item
  Create → Kayıt
\end{enumerate}

\begin{center}\rule{0.5\linewidth}{0.5pt}\end{center}

\subsection{Faz 2 — Sensörler (Sensor-Based
Things)}\label{faz-2-sensuxf6rler-sensor-based-things}

\subsubsection{2.1 SeismicSensor: “Kadıköy Sismik Sensör
001”}\label{seismicsensor-kadux131kuxf6y-sismik-sensuxf6r-001}

\begin{longtable}[]{@{}ll@{}}
\toprule\noalign{}
Alan & Değer \\
\midrule\noalign{}
\endhead
\bottomrule\noalign{}
\endlastfoot
\textbf{Thing Type} & \texttt{sensor} \\
\textbf{DTDL Interface} & \texttt{dtmi:iodt2:SeismicSensor;1} \\
\textbf{ID} & \texttt{kadikoy-seismic-001} \\
\textbf{Name} & Kadıköy Sismik Sensör 001 \\
\textbf{Manufacturer} & Güralp Systems \\
\textbf{Model} & CMG-5TD \\
\textbf{Serial Number} & GS-2026-00451 \\
\end{longtable}

\textbf{Adımlar:}

\begin{enumerate}
\def\labelenumi{\arabic{enumi}.}
\tightlist
\item
  Thing Type: \textbf{Sensor}
\item
  DTDL Interface: \textbf{Seismic Sensor} → Auto-Fill
\item
  Domain metadata alanlarını doldur: Manufacturer, Model, Serial Number,
  Firmware Version (\texttt{v3.2.1})
\item
  Auto-Fill sonrası property listesi:
\end{enumerate}

\begin{longtable}[]{@{}llll@{}}
\toprule\noalign{}
Property & Type & Açıklama & Writable \\
\midrule\noalign{}
\endhead
\bottomrule\noalign{}
\endlastfoot
\texttt{sensorType} & string (Enum) & \texttt{accelerometer} & No \\
\texttt{sensitivity} & float & 0.001 & Yes \\
\texttt{samplingRate} & float & 200 Hz (SensorTwin’den miras) & Yes \\
\texttt{accuracy} & float & 0.01 (SensorTwin’den miras) & No \\
\texttt{accelerationX} & float & Telemetry — m/s² & No \\
\texttt{accelerationY} & float & Telemetry — m/s² & No \\
\texttt{accelerationZ} & float & Telemetry — m/s² & No \\
\texttt{peakGroundAcceleration} & float & Telemetry — m/s² & No \\
\texttt{magnitude} & float & Telemetry & No \\
\texttt{frequency} & float & Telemetry — Hz & No \\
\end{longtable}

\begin{enumerate}
\def\labelenumi{\arabic{enumi}.}
\setcounter{enumi}{4}
\tightlist
\item
  Commands: \texttt{calibrate}, \texttt{triggerAlert}
\item
  \textbf{Relationships} sekmesi: \texttt{monitoredBy} → target:
  \texttt{kadikoy-hastanesi} (Building ile ilişkilendir)
\item
  Location: Hastane binasının yanına konumla
\item
  YAML Preview’da kontrol et → Create
\end{enumerate}

\begin{center}\rule{0.5\linewidth}{0.5pt}\end{center}

\subsubsection{2.2 TemperatureSensor: “Hastane İç Sıcaklık
Sensörü”}\label{temperaturesensor-hastane-iuxe7-sux131caklux131k-sensuxf6ruxfc}

\begin{longtable}[]{@{}ll@{}}
\toprule\noalign{}
Alan & Değer \\
\midrule\noalign{}
\endhead
\bottomrule\noalign{}
\endlastfoot
\textbf{Thing Type} & \texttt{sensor} \\
\textbf{DTDL Interface} & \texttt{dtmi:iodt2:TemperatureSensor;1} \\
\textbf{ID} & \texttt{hastane-temp-001} \\
\textbf{Name} & Hastane İç Sıcaklık Sensörü \\
\textbf{Manufacturer} & Bosch \\
\textbf{Model} & BME280 \\
\end{longtable}

\textbf{Adımlar:}

\begin{enumerate}
\def\labelenumi{\arabic{enumi}.}
\tightlist
\item
  Thing Type: \textbf{Sensor} → DTDL: \textbf{Temperature Sensor} →
  Auto-Fill
\item
  Property’ler:
\end{enumerate}

\begin{longtable}[]{@{}llll@{}}
\toprule\noalign{}
Property & Type & Value & Writable \\
\midrule\noalign{}
\endhead
\bottomrule\noalign{}
\endlastfoot
\texttt{temperatureUnit} & string (Enum) & \texttt{celsius} & Yes \\
\texttt{alertThreshold} & float & 45.0 & Yes \\
\texttt{temperature} & float & Telemetry — °C & No \\
\end{longtable}

\begin{enumerate}
\def\labelenumi{\arabic{enumi}.}
\setcounter{enumi}{2}
\tightlist
\item
  Location: Hastane binası içi
\item
  Create
\end{enumerate}

\begin{center}\rule{0.5\linewidth}{0.5pt}\end{center}

\subsubsection{2.3 PM25Sensor: “Kadıköy Hava Kalitesi
İstasyonu”}\label{pm25sensor-kadux131kuxf6y-hava-kalitesi-istasyonu}

\begin{longtable}[]{@{}ll@{}}
\toprule\noalign{}
Alan & Değer \\
\midrule\noalign{}
\endhead
\bottomrule\noalign{}
\endlastfoot
\textbf{Thing Type} & \texttt{sensor} \\
\textbf{DTDL Interface} & \texttt{dtmi:iodt2:PM25Sensor;1} \\
\textbf{ID} & \texttt{kadikoy-pm25-001} \\
\textbf{Name} & Kadıköy PM2.5 Hava Kalitesi Sensörü \\
\textbf{Domain} & \texttt{air\_quality} \\
\end{longtable}

\textbf{Adımlar:}

\begin{enumerate}
\def\labelenumi{\arabic{enumi}.}
\tightlist
\item
  Thing Type: \textbf{Sensor} → DTDL: \textbf{PM2.5 Sensor} → Auto-Fill
\item
  Property’ler:
\end{enumerate}

\begin{longtable}[]{@{}llll@{}}
\toprule\noalign{}
Property & Type & Value & Writable \\
\midrule\noalign{}
\endhead
\bottomrule\noalign{}
\endlastfoot
\texttt{aqi} & integer & Calculated — AQI Index & No \\
\texttt{aqiCategory} & string (Enum) & \texttt{good} / \texttt{moderate}
/ … & No \\
\texttt{alertThreshold} & float & 35.0 (µg/m³) & Yes \\
\texttt{pm25} & float & Telemetry — µg/m³ & No \\
\end{longtable}

\begin{enumerate}
\def\labelenumi{\arabic{enumi}.}
\setcounter{enumi}{2}
\tightlist
\item
  Location: Bağdat Caddesi üzerinde
\item
  Create
\end{enumerate}

\begin{quote}
\textbf{Not:} Bu sensör \textbf{farklı bir domain} (air\_quality) —
senaryoda cross-domain izleme kapasitesini gösterir.
\end{quote}

\begin{center}\rule{0.5\linewidth}{0.5pt}\end{center}

\subsection{Faz 3 — Cihazlar (Device-Based
Things)}\label{faz-3-cihazlar-device-based-things}

\subsubsection{3.1 WeatherStation: “Kadıköy Meteoroloji
İstasyonu”}\label{weatherstation-kadux131kuxf6y-meteoroloji-istasyonu}

\begin{longtable}[]{@{}ll@{}}
\toprule\noalign{}
Alan & Değer \\
\midrule\noalign{}
\endhead
\bottomrule\noalign{}
\endlastfoot
\textbf{Thing Type} & \texttt{device} \\
\textbf{DTDL Interface} & \texttt{dtmi:iodt2:WeatherStation;1} \\
\textbf{ID} & \texttt{kadikoy-weather-001} \\
\textbf{Name} & Kadıköy Meteoroloji İstasyonu \\
\end{longtable}

\textbf{Adımlar:}

\begin{enumerate}
\def\labelenumi{\arabic{enumi}.}
\tightlist
\item
  Thing Type: \textbf{Device} → DTDL: \textbf{Weather Station} →
  Auto-Fill
\item
  DTDL özeti gösterir: \textbf{2 component} (temperatureSensor,
  humiditySensor), 4 telemetry, 2 command
\item
  \textbf{Bu component-based bir cihaz} — DTDL’de iç içe sensör
  barındırır
\item
  Auto-Fill sonrası:
\end{enumerate}

\begin{longtable}[]{@{}llll@{}}
\toprule\noalign{}
Property & Type & Açıklama & Writable \\
\midrule\noalign{}
\endhead
\bottomrule\noalign{}
\endlastfoot
\texttt{pressure} & float & Telemetry — Pa & No \\
\texttt{windSpeed} & float & Telemetry — m/s & No \\
\texttt{windDirection} & integer & Telemetry — 0-360° & No \\
\texttt{rainfall} & float & Telemetry — mm & No \\
\end{longtable}

\begin{enumerate}
\def\labelenumi{\arabic{enumi}.}
\setcounter{enumi}{4}
\tightlist
\item
  Commands: \texttt{startDataCollection}, \texttt{stopDataCollection}
\item
  \textbf{Relationships} sekmesi:

  \begin{itemize}
  \tightlist
  \item
    \texttt{installedAt} → target: \texttt{kadikoy-hastanesi} (Building)
  \item
    \texttt{monitorsStreet} → target: \texttt{bagdat-caddesi-seg-a}
    (Street)
  \end{itemize}
\item
  Location: Hastane çatısı
\item
  Create
\end{enumerate}

\begin{quote}
\textbf{Demo Notu:} Bu Thing, component-based modelemeyi gösterir.
İçindeki \texttt{temperatureSensor} ve \texttt{humiditySensor}
component’ları DTDL katmanında tanımlıdır.
\end{quote}

\begin{center}\rule{0.5\linewidth}{0.5pt}\end{center}

\subsubsection{3.2 BaseStation: “Kadıköy 5G Baz
İstasyonu”}\label{basestation-kadux131kuxf6y-5g-baz-istasyonu}

\begin{longtable}[]{@{}ll@{}}
\toprule\noalign{}
Alan & Değer \\
\midrule\noalign{}
\endhead
\bottomrule\noalign{}
\endlastfoot
\textbf{Thing Type} & \texttt{device} \\
\textbf{DTDL Interface} & \texttt{dtmi:iodt2:BaseStation;1} \\
\textbf{ID} & \texttt{kadikoy-5g-bs-001} \\
\textbf{Name} & Kadıköy 5G Baz İstasyonu \\
\end{longtable}

\textbf{Adımlar:}

\begin{enumerate}
\def\labelenumi{\arabic{enumi}.}
\tightlist
\item
  Thing Type: \textbf{Device} → DTDL: \textbf{Base Station} → Auto-Fill
\item
  Property’ler:
\end{enumerate}

\begin{longtable}[]{@{}llll@{}}
\toprule\noalign{}
Property & Type & Value & Writable \\
\midrule\noalign{}
\endhead
\bottomrule\noalign{}
\endlastfoot
\texttt{stationId} & string & \texttt{KDK-5G-001} & No \\
\texttt{technology} & string (Enum) & \texttt{5g} & No \\
\texttt{coverageRadius} & float & 500.0 (metre) & No \\
\texttt{hasBackupPower} & boolean & \texttt{true} & No \\
\texttt{priorityLevel} & string (Enum) & \texttt{emergency} & Yes \\
\texttt{signalStrength} & float & Telemetry — dB & No \\
\texttt{activeConnections} & integer & Telemetry & No \\
\texttt{powerLevel} & float & Telemetry — \% & No \\
\texttt{networkLoad} & float & Telemetry — \% & No \\
\end{longtable}

\begin{enumerate}
\def\labelenumi{\arabic{enumi}.}
\setcounter{enumi}{2}
\tightlist
\item
  Commands: \texttt{activateEmergencyMode}, \texttt{boostSignal},
  \texttt{switchToBackupPower}
\item
  \textbf{Relationships:}

  \begin{itemize}
  \tightlist
  \item
    \texttt{covers} → target: \texttt{kadikoy-hastanesi} (Building)
  \item
    \texttt{coversStreet} → target: \texttt{bagdat-caddesi-seg-a}
    (Street)
  \end{itemize}
\item
  Location: Bağdat Caddesi yakını
\item
  Create
\end{enumerate}

\begin{center}\rule{0.5\linewidth}{0.5pt}\end{center}

\subsection{Faz 4 — Doğrulama ve
Sorgulama}\label{faz-4-doux11frulama-ve-sorgulama}

\subsubsection{4.1 Thing Listesi
Kontrolü}\label{thing-listesi-kontroluxfc}

\textbf{Things} sayfasına git → 7 Thing kaydı görünür:

\begin{longtable}[]{@{}
  >{\raggedright\arraybackslash}p{(\linewidth - 8\tabcolsep) * \real{0.1000}}
  >{\raggedright\arraybackslash}p{(\linewidth - 8\tabcolsep) * \real{0.2333}}
  >{\raggedright\arraybackslash}p{(\linewidth - 8\tabcolsep) * \real{0.2000}}
  >{\raggedright\arraybackslash}p{(\linewidth - 8\tabcolsep) * \real{0.2667}}
  >{\raggedright\arraybackslash}p{(\linewidth - 8\tabcolsep) * \real{0.2000}}@{}}
\toprule\noalign{}
\begin{minipage}[b]{\linewidth}\raggedright
\#
\end{minipage} & \begin{minipage}[b]{\linewidth}\raggedright
Thing
\end{minipage} & \begin{minipage}[b]{\linewidth}\raggedright
Type
\end{minipage} & \begin{minipage}[b]{\linewidth}\raggedright
Domain
\end{minipage} & \begin{minipage}[b]{\linewidth}\raggedright
DTDL
\end{minipage} \\
\midrule\noalign{}
\endhead
\bottomrule\noalign{}
\endlastfoot
1 & Kadıköy Devlet Hastanesi & Component & seismic & Building;1 \\
2 & Bağdat Caddesi - Segment A & Component & seismic & Street;1 \\
3 & Kadıköy Sismik Sensör 001 & Sensor & seismic & SeismicSensor;1 \\
4 & Hastane İç Sıcaklık Sensörü & Sensor & environmental &
TemperatureSensor;1 \\
5 & Kadıköy PM2.5 Sensörü & Sensor & air\_quality & PM25Sensor;1 \\
6 & Kadıköy Meteoroloji İstasyonu & Device & environmental &
WeatherStation;1 \\
7 & Kadıköy 5G Baz İstasyonu & Device & seismic & BaseStation;1 \\
\end{longtable}

\subsubsection{4.2 Detay Sayfası Demo}\label{detay-sayfasux131-demo}

\texttt{Kadıköy\ Sismik\ Sensör\ 001} detayına gir:

\begin{itemize}
\tightlist
\item
  \textbf{Properties} sekmesi: 10 property (6 telemetry + 4 config)
\item
  \textbf{Relationships} sekmesi:
  \texttt{monitoredBy\ →\ kadikoy-hastanesi}
\item
  \textbf{Commands} sekmesi: \texttt{calibrate}, \texttt{triggerAlert}
\item
  \textbf{DTDL Binding} sekmesi: \texttt{dtmi:iodt2:SeismicSensor;1}
  bağlantısı, writable/unit badge’leri
\end{itemize}

\subsubsection{4.3 SPARQL Sorguları}\label{sparql-sorgularux131}

\textbf{Search Things} sayfasında SPARQL moduna geç:

\textbf{Sorgu 1 — Tüm Sensörleri Listele:}

\begin{Shaded}
\begin{Highlighting}[]
\NormalTok{SELECT ?name ?type WHERE \{}
\NormalTok{  ?s a ts:TwinInterface .}
\NormalTok{  ?s ts:name ?name .}
\NormalTok{  ?s ts:thingType "sensor" .}
\NormalTok{\}}
\end{Highlighting}
\end{Shaded}

\textbf{Sorgu 2 — Deprem Domain’indeki Tüm Thing’ler:}

\begin{Shaded}
\begin{Highlighting}[]
\NormalTok{SELECT ?name ?thingType WHERE \{}
\NormalTok{  ?s a ts:TwinInterface .}
\NormalTok{  ?s ts:name ?name .}
\NormalTok{  ?s ts:thingType ?thingType .}
\NormalTok{  ?s ts:dtdlCategory "seismic" .}
\NormalTok{\}}
\end{Highlighting}
\end{Shaded}

\textbf{Sorgu 3 — Hastane ile İlişkili Tüm Thing’ler:}

\begin{Shaded}
\begin{Highlighting}[]
\NormalTok{SELECT ?name ?relName WHERE \{}
\NormalTok{  ?s a ts:TwinInstance .}
\NormalTok{  ?s ts:name ?name .}
\NormalTok{  ?s ts:hasInstanceRelationship ?rel .}
\NormalTok{  ?rel ts:targetInterface "kadikoy{-}hastanesi" .}
\NormalTok{  ?rel ts:name ?relName .}
\NormalTok{\}}
\end{Highlighting}
\end{Shaded}

\subsubsection{4.4 Export Demo}\label{export-demo}

Herhangi bir Thing’in detay sayfasından \textbf{“Export ZIP”} butonuna
tıkla → İndirilen ZIP içinde:

\begin{itemize}
\tightlist
\item
  \texttt{TwinInterface.yaml} — Blueprint tanımı
\item
  \texttt{TwinInstance.yaml} — Concrete instance tanımı
\end{itemize}

\begin{center}\rule{0.5\linewidth}{0.5pt}\end{center}

\subsection{Senaryo Akış
Diyagramı}\label{senaryo-akux131ux15f-diyagramux131}

\begin{verbatim}
┌─────────────────────────────────────────────────────────┐
│                    FAZ 1: ALTYAPI                       │
│                                                         │
│  ┌──────────────┐         ┌──────────────────────┐      │
│  │   Building   │         │       Street         │      │
│  │  (Component) │         │    (Component)       │      │
│  │  Hastane     │         │  Bağdat Cad.         │      │
│  └──────┬───────┘         └──────────┬───────────┘      │
│         │                            │                  │
├─────────┼────────────────────────────┼──────────────────┤
│         │       FAZ 2: SENSÖRLER     │                  │
│         │                            │                  │
│  ┌──────▼───────┐  ┌────────────┐  ┌▼──────────────┐   │
│  │SeismicSensor │  │ TempSensor │  │  PM2.5 Sensor │   │
│  │  (Sensor)    │  │  (Sensor)  │  │   (Sensor)    │   │
│  │  seismic     │  │  environ.  │  │  air_quality  │   │
│  └──────┬───────┘  └─────┬──────┘  └───────────────┘   │
│         │                │                              │
├─────────┼────────────────┼──────────────────────────────┤
│         │     FAZ 3: CİHAZLAR       │                  │
│         │                            │                  │
│  ┌──────▼───────────────┐  ┌────────▼──────────────┐   │
│  │   WeatherStation     │  │    BaseStation        │   │
│  │   (Device)           │  │    (Device)           │   │
│  │   [temp + humidity]  │  │    5G Emergency       │   │
│  │   component-based    │  │    Communication      │   │
│  └──────────────────────┘  └───────────────────────┘   │
│                                                         │
├─────────────────────────────────────────────────────────┤
│  FAZ 4: DOĞRULAMA                                       │
│  - Thing Listesi (7 kayıt)                              │
│  - Detay & DTDL Binding kontrolü                        │
│  - SPARQL sorguları (3 örnek)                           │
│  - ZIP Export                                           │
└─────────────────────────────────────────────────────────┘
\end{verbatim}

\begin{center}\rule{0.5\linewidth}{0.5pt}\end{center}

\subsection{Demo Süresince Vurgulanan
Özellikler}\label{demo-suxfcresince-vurgulanan-uxf6zellikler}

\begin{longtable}[]{@{}
  >{\raggedright\arraybackslash}p{(\linewidth - 2\tabcolsep) * \real{0.3214}}
  >{\raggedright\arraybackslash}p{(\linewidth - 2\tabcolsep) * \real{0.6786}}@{}}
\toprule\noalign{}
\begin{minipage}[b]{\linewidth}\raggedright
Özellik
\end{minipage} & \begin{minipage}[b]{\linewidth}\raggedright
Gösterildiği Adım
\end{minipage} \\
\midrule\noalign{}
\endhead
\bottomrule\noalign{}
\endlastfoot
\textbf{DTDL Interface seçimi \& Auto-Fill} & Her Thing oluşturmada \\
\textbf{3 Thing Type farkı} (sensor/device/component) & Faz 1, 2, 3 \\
\textbf{Component-based modelleme} & WeatherStation (iç içe
sensörler) \\
\textbf{Cross-domain izleme} & PM2.5 (air\_quality) + SeismicSensor
(seismic) aynı bölgede \\
\textbf{Relationship tanımlama} & SeismicSensor→Building,
WeatherStation→Building, BaseStation→Street \\
\textbf{Konum ve reverse geocoding} & Her Thing’de harita üzerinden
konum seçimi \\
\textbf{Canlı YAML Preview} & Her form değişikliğinde sağ panelde \\
\textbf{DTDL Validation} & Her DTDL-bound Thing’de yeşil tik \\
\textbf{SPARQL ile sorgulama} & Faz 4’te 3 farklı sorgu \\
\textbf{ZIP Export} & Faz 4’te herhangi bir Thing \\
\end{longtable}

\end{document}
